%-------------------------------------------------------------------------------
%	SECTION TITLE
%-------------------------------------------------------------------------------
\cvsection{Projects}


%-------------------------------------------------------------------------------
%	CONTENT
%-------------------------------------------------------------------------------
\begin{cventries}

%---------------------------------------------------------
\cventry
  {\url{https://upbri.gitlab.io/biogrowleR/}} % Link to the website
  {biogrowleR} % Name of Project
  {}{}
  {
    \begin{cvitems} % Description(s) of tasks/responsibilities
      \item {An R package that provides tutorials and auxiliary 
             functions on how to analyse growth measurement data.}
      \item {Package was taught in workshops and has been used by
            other life scientist.}
    \end{cvitems}
  }


%---------------------------------------------------------
\cventry
  {\url{https://gitlab.com/upbri/ttmap}} % Link to the website
  {ttmap} % Name of Project
  {}{}
  {
    \begin{cvitems} % Description(s) of tasks/responsibilities
      \item {R package implementing Two-Tier Mapper, a 
             topological tool to analyse RNA-Seq data.}
      \item {Developed a R package with a more user-friendly interface.}
      \item {Supervised a student that added tests and a shiny app interface
            to the package.} 
    \end{cvitems}
  }



%---------------------------------------------------------
\cventry
  {\url{https://github.com/chronchi/ProteinPersistent.jl}} % Link to the repo
  {ProteinPersistent.jl} % Name of Project
  {}{}
  {
    \begin{cvitems} % Description(s) of tasks/responsibilities
      \item {Package that provides an interface for some functions of BioPython.
      It also calculates the persistent homology of a protein using the python
      package ripser.}
    \end{cvitems}
  }

%---------------------------------------------------------
\cventry
  {\url{https://github.com/chronchi/HSP.jl}} % Link to the repo
  {HSP.jl} % Name of Project
  {}{}
  {
    \begin{cvitems} % Description(s) of tasks/responsibilities
      \item {Julia implementation of a package to calculate the
      optimal Hansen Solubility Parameters.}
    \end{cvitems}
  }

%---------------------------------------------------------
\cventry
  {\url{https://github.com/chronchi/MapperMDS.jl}} % Link to the repo
  {MapperMDS.jl} % Name of Project
  {}{}
  {
    \begin{cvitems} % Description(s) of tasks/responsibilities
      \item {Mapper is an algorithm from topological data analysis that
      helps visualize high dimensional data. This is
      an implementation in Julia that particularly accepts a distance matrix
      as input.}
    \end{cvitems}
  }

%---------------------------------------------------------
\cventry
  {\url{https://github.com/chronchi/PersistenceImage.jl}} % Link to the repo
  {PersistenceImage.jl} % Name of Project
  {}{}
  {
    \begin{cvitems} % Description(s) of tasks/responsibilities
      \item {Persistence image is a vectorization method for persistence
      diagrams. This is an implementation of the algorithm in Julia.}
    \end{cvitems}
  }

%---------------------------------------------------------
\cventry
  {\url{https://github.com/chronchi/perscode}} % Link to the repo
  {perscode} % Name of Project
  {}{}
  {
    \begin{cvitems} % Description(s) of tasks/responsibilities
      \item {Perscode is a vectorization method for persistence
      diagrams. This is an implementation of the algorithm in python.}
    \end{cvitems}
  }

%---------------------------------------------------------
\cventry
  {\url{https://github.com/chronchi/3dPD}} % Link to the repo
  {3dPD} % Name of Project
  {}{}
  {
    \begin{cvitems} % Description(s) of tasks/responsibilities
        \item {Visualization tool for optimal cycles (w.r.t. number of edges) and persistence
	       diagrams of three-dimensional datasets.}
    \end{cvitems}
  }

%---------------------------------------------------------
\end{cventries}
