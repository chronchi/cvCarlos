%------------------------------------------------------------------------------
%	SECTION TITLE
%-------------------------------------------------------------------------------
\cvsection{Research Experience}


%-------------------------------------------------------------------------------
%	CONTENT
%-------------------------------------------------------------------------------



\begin{cventries}

  \cventry
    {Marie Curie fellow (MSCA - Horizon 2020)}
    {PhD student at EPFL}
    {Lausanne, Switzerland} 
    {Sep. 2020 - Current} 
    {
      \begin{cvitems} 
        \item {Developed a novel platform (EMBER) for biomarker discovery in breast
               cancer using bulk RNA-seq and microarray}
        \item {Developed several R packages for biostatistics and bioinformatic
                analysis in the lab (biogrowleR), increasing the productivity of
                team members and standardizing several protocols} 
        \item {Used Bayesian Inference in novel ways to extract insights from 
               data generated in the lab, leading to better
               interpretation of the results}
        \item { Developed several analysis pipelines for transcriptomics
                data that is shared in the group and used by non-bioinformaticians as well}
        \item {Experienced in communicating with bioscientists. Paying attention to their
                needs and helping them in clarifying their questions with biostatistics 
                and computational biology (biogrowleR)}
      \end{cvitems} 
    }
%-------------------------------------------------------------------------------


\begin{cventries}

  \cventry
    {funded by MSCA - Horizon 2020}
    {Research intern at AstraZeneca}
    {Cambridge, UK} 
    {Nov. 2022 - Nov. 2022} 
    {
      \begin{cvitems} 
        \item {Academic placement to learn and understand the research 
            process in
            pharmaceutical companies. Presented my research for
            multiple scientists in both the bioscience and bioinformatics
            department}
        \item {Introduced novel ways to interpret their drug pipeline discovery platform
                results by using bayesian inference}
      \end{cvitems} 
    }
    %-------------------------------------------------------------------------------

\begin{cventries}

  \cventry
    {funded by École polytechnique fédérale de Lausanne (EPFL)}
    {Research assistant at Brisken's lab at School of Life Sciences - EPFL}
    {Lausanne, Switzerland} 
    {Jan. 2020 - August 2020} 
    {
      \begin{cvitems} 
        \item {Improved current packaging standards in the laboratory and 
                introduced software best practices on the way (ttmap)}
        \item {Studied statistical methods and how they are applied
            to answer breast cancer related questions}
      \end{cvitems} 
    }
%-------------------------------------------------------------------------------
  \cventry
    {funded by Fundação de Amparo à Pesquisa do Estado de São Paulo} % Funding agency
    {Master's student} % Job title
    {São Carlos, Brazil} % Location
    {Aug. 2017 - Nov. 2020} % Date(s)
    {
      \begin{cvitems} % Description(s) of tasks/responsibilities
        \item {Developed novel ways to predict protein stability by 
                applying persistent homology, tools from topological
                data analysis (TDA) and deep learning}
        \item {Combined machine learning and persistent homology to improve
               the accuracy in image classification problems}
        \item {Implemented TDA algorihtms in Julia (MapperMDS.jl, PersistenceImage.jl),
               allowing researchers to speed up their computations by using a faster
               language}
        \item {Developed web-based tools to visualize the results of optimal cycle
               extraction from persistent homology, helping researchers understand
               their results (3dPD)}
        \item {Collaboration with material science researchers to understand the
               optimal Hansen Solubility Parameters of organic semiconductors (HSP.jl)}
      \end{cvitems}
    }

%---------------------------------------------------------
\cventry
  {funded by Programa de Atração de Jovens Talentos (CSF-PAJT)} % Funding agency
  {Undergradute researcher} % Job title
  {Curitiba, Brazil} % Location
  {Jul. 2016 - Jul. 2017} % Date(s)
  {
    \begin{cvitems} % Description(s) of tasks/responsibilities
        \item {Applied Convolutional Neural Networks (CNNs) to predict LaTeX 
             characters, improving the accuracy of existing algorithms}
    \end{cvitems}
  }

%---------------------------------------------------------
\end{cventries}
